\begin{itemize}
    \item RMSE / MSE
    \item R-Squared
    \item Residual analysis (normality and correlation, heteroschedasticity)
    \item Mean Absolute Error
    \item  Rolling window approach
\end{itemize}

Interesting things to mention:
\begin{itemize}
	\item Check difference in portfolio returns by starting daily rolling window with 1 or 4/5 years of training especially consider the period around 2008-2010. This shows how the model benefits from not having training data that is too different from a period of extreme market volatility. With 1 year of training and a daily re-balancing frequency the model is much more free to adapt to extreme market conditions. (This period is in fact the period of greater volatility, see Figure \ref{fig:stock-std-plot})
	(Look at the bearish dip around 2008-2010). This means that this model needs two memories: a long and a short memory, the long one should be correctly weighted. In this model this is happening manually when I decide with how much training data to start the rolling window approach.
	
	Lot of first years of training --> old memory weighs more
	1/2 years of first training --> new memory weighs more faster
	(Connect this with LSTM Neural Network)
\end{itemize}