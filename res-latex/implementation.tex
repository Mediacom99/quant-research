
Explain from formattedData to portfolio results. With rolling window. 

\begin{enumerate}
\item divide returns and factors datasets into training and testing
\item Normalize factors
\item Apply PCA on factors
\item Lag factors by shifting them one day into the future
\item run SGD regressor
\item calculate cov matrix with formula with eq. \eqref{cov-matrix-returns}
\item Find optimized portfolio weights using cov matrix 
\item calculate portfolio performance with this optimized weights on the testing data.
\end{enumerate}
In order to better represent an actual real-world implementation of this model I decided to use a Rolling Window approach. I start with a certain division between training and testing data, I run the aformentioned algorithm, then I add the testing data used in this run to the training data that will be used in the next run. Then I repeat. 

\begin{algorithm}
	\caption{Rolling Window Model Deployment}
	\begin{algorithmic}[1]
		\STATE TEMPDATE = initial training/testing division date
		\WHILE{TEMPDATE < FINALDATE}
		\STATE Divide returns and factors into training and testing initial datasets
		\STATE Normalize the factors
		\STATE Apply two PCA runs on factors
		\STATE Lag factors
		\STATE Run SGD regressor
		\STATE calculate calculate cov matrix with formula with eq. \eqref{cov-matrix-returns}
		\STATE Find optimized portfolio weights using cov matrix
		\STATE Calculate portfolio performance over period
		\STATE TEMPDATE + chosen OFFSET (day, week, month, year)
		\ENDWHILE
		\RETURN daily portfolio returns over every testing period in one single dataframe
	\end{algorithmic}
\end{algorithm}