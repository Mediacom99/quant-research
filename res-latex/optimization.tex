Since Markov first economic theories there have been developed many ways to optimize the allocation of different assets in a portfolio. Should one consider the historical average of the returns ? Should one try to find the portfolio with the highest return ? Or with the lowest standard deviation ? Or maybe one should allocate assets in a portfolio in order to reach the minimum possible variance. For this task I decided to allocate the risk instead of allocating the capital. I used a risk-parity approach that can be summed up like this: the portfolio is optimized such that each asset's contribution to the total portfolio volatility is equal. I followed the procedure in the third section of \cite{risk-parity-hard}.

Before writing the objective function of the optimization problem, two quantities need to be introduced: the marginal risk contribution (MRC) and the risk contribution (RC). The MRC is calculated as follows:
\begin{equation}
	\vec{\textit{MRC}} = \mathbf{\Sigma}\vec{w}
\end{equation}
where $\Sigma$ is the covariance matrix of expected returns. The components of MRC represent the contribution of each asset to the portfolio volatility based on the asset's weight and the asset's volatility.

The Risk Contribution then is:
\begin{equation}
	\vec{\textit{RC}} = \frac{\vec{w} * (\vec{\textit{MRC}})^T}{\Vec{w}^{T}\mathbf{\Sigma}\Vec{w}}
\end{equation}
where $*$ represents element-wise multiplication and the denominator is the total portfolio volatility as in eq.\eqref{portfolio-distribution}.
Now the objective function can be constructed easily, we impose that each element of the Risk Contribution vector is equal, thus:
\begin{equation}
	f(\vec{x}) = \sum_{\substack{i = 1 \\ j = i + 1}}^{N} (\vec{\textit{RC}}_i (\vec{x}) - \vec{\textit{RC}}_j (\vec{x}))^2 
	\label{risk-parity-objfunc}
\end{equation}
So the minimization problem, adding the constraints for a long-only trading model and complete asset allocation is:

\begin{equation}
	\begin{cases}
		&\text{minimize }  f(\vec{x}) \nonumber \\
		& \text{subject\ to:}\\
		& \sum_{i=1}^N x_i = 1 \nonumber \\
		&\vec{x} \in \mathcal{R}^N_{+} \nonumber \\
	\end{cases}
	\label{V}
\end{equation}


where $\vec{x}$ is the portfolio weights vector and $N$ is number of assets in the portfolio. I used the function \textit{scipy.optimize.minimize} with the Sequential Least Squares Programming method.

The solutions to this minimization problem are the optimal, risk-parity adjusted, weights with which to build the portfolio.

The optimization procedure has been implemented in the custom python module \textbf{optimize.py}.

\clearpage