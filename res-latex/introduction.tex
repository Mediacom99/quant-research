


This is the report for the task assigned to me by the team at Minerva Technologies. The purpose of this task is to develop a long-only trading model of five stock indices resulting in a portfolio weight matrix. The input data is comprised of daily historical logarithmic returns from the 30th September 2003 to the 31st December 2019 for the five stock indices, as well as a series of other factors: rates, commodities, Forex currency crossings. Macroeconomic indices of the stock indices' countries and fundamentals of the stock were also given.

Given the amount of data and task's objective I decided to use a simple but effective model: the portfolio will be optimized using a risk-parity approach were the allocation of assets is determined by imposing that each asset's contribution to the portfolio total variance is equal. In order to find the covariance matrix of expected returns based on the historical data, necessary to optimize the portfolio, I decided to use a linear multi-factor model combined with Principal Component Analysis on the factors in order to reduce their dimensionality but also eliminate linear correlation, grouping them by how much they contribute to the overall variation of the factors data.

The model will be tested by implementing a rolling window approach with a weekly rebalancing frequency. For each week the model will be retrained and the weights recalculated in order to calculate the portfolio weight matrix and test the portfolio return and volatility. A K-Fold Cross Validation is used on the training data in order to choose the best linear regressor based on the their performance scores.


Even though normality is required only for the residuals of the linear regression, having data that is approximately normally distributed makes calculation easier since most of the information is captured by the first two moments: mean and variance. For this reason Section \ref{dataanalysis} is dedicated to testing data for its normality as well as for stationarity and cross-asset relationships to better understand the nature of the data and better interpret the result of the model.
